\documentclass[letterpaper,12pt]{article}
\usepackage[utf8]{inputenc}
\pagestyle{plain}
\usepackage{graphicx}
\usepackage[small]{caption}
\usepackage{cite}
\usepackage{amsmath}
\usepackage{multirow}
\usepackage{setspace}
\usepackage{xcolor}
\usepackage{authblk}
\usepackage{url}
\usepackage{hyperref}

\pagenumbering{roman}
%Insert your title and author list below
\title{ePIC Exclusive, Diffractive and Tagging Analysis Note Template}
\author[1]{L. Boi}
\author[2]{S. McDuck}

\affil[1]{School of Bread, Seeds and Waterfowl, University of Yolk, UK}
\affil[2]{School of Mallard Management, Duck University, NC, USA}

\date{MONTH YEAR}

\begin{document}

\maketitle
\begin{abstract}
Abstract goes here
\end{abstract}

\begin{figure}[h]
    \centering
    \includegraphics[scale=0.5]{Figures/EPIC-logo_black.png}
\end{figure}

\pagebreak
\tableofcontents

\pagebreak
\pagenumbering{arabic}

\section{Introduction}\label{sec:Intro}

Some background info to your channel. Brief overview of the physics interest in the channel.
%Include figures etc as needed
\begin{figure}[h]
    \centering
    \includegraphics[scale=0.5]{Figures/EPIC-longboi.png}
    \caption{A re-imagined ePIC logo incorporating \href{https://www.york.ac.uk/about/history/long-boi/}{York's most famous duck.}}
\label{fig:LBoi}
\end{figure}

\section{Simulation Overview}\label{sec:Sim_Overview}

Details on the simulation, event generator used, detector geometry, beam conditions etc. Details on the relevant simulation campaign. Reference other sections via Sec.~\ref{sec:Intro} and cite things via \cite{jr:2022_YR}. Reference figures as Fig.~\ref{fig:LBoi}.

\subsection{Event Generator Details}\label{subsec:EvGen}

Information on event generator utilised. Specify version, link to instructions for running code (should be provided for Production WG). Break out other components into subsections if desired/needed (e.g. if simulation geometry is specific/unique).

\section{Event Selection}\label{sec:EvSelect}

Information on your event selection procedure/process. Highlight any specific subsystems used if relevant. Outline cuts used, ordering, rationale. If a cut is ``unusual'' or non-standard, make sure you discuss it. When showing kinematic variables, \textbf{clarify the reconstruction method.} If needed, include a description of the reconstruction method.

\subsection{Analysis Code}\label{subsec:Analysis_Code}

Information on where to find your analysis code, how to run it. Consider adding additional details to the appendix (weird compilation quirks etc).

\section{Results and Discussion}\label{sec:Results_Discuss}

Your results with key performance plots etc. Consult checklist of key figures, see \href{https://docs.google.com/presentation/d/1bqz9_GPvPoW4oz1m8KvzuUhPJZBe_CfU5APMt0LjfaU/edit?slide=id.g3338e3f4b69_0_51#slide=id.g3338e3f4b69_0_51}{page three of this presentation} as an example of plots to include.

\pagebreak
\appendix
\section{Appendix}
\label{Appendix:tRes}

As discussed in Sec.~\ref{sec:statistics}, the resolution in $t$ varies with beam energy and across $Q^{2}$ bins, as shown in Figs.~\ref{fig:tRes_10x130_Q2_5_6}--\ref{fig:tRes_10x250_Q2_32_33}. For $10\times130$ events, the resolution in $t$ is comfortably within the chosen width of each bin for the range of $t$ that is used in the form factor determination. The exception to this is for high $-t$ values in higher $Q^{2}$ bins, as can be seen in Fig.~\ref{fig:tRes_10x130_Q2_32_33}. As such, for the high $Q^{2}$ bins at this beam energy, either broader $t$ bins must be used, or a bin-migration systematic will need to be applied. 

\begin{figure}[h]
    \centering
    \includegraphics[width=0.495\textwidth]{Figures/Appendix_Figs/10on130_tRes_2D_5_6_Q2.png}
    \includegraphics[width=0.495\textwidth]{Figures/Appendix_Figs/10on130_tRes_5_6_Q2}
    \caption{(Left) $\Delta t = t_{mc} - t_{eXBABE}$ as a function of $-t_{MC}$ for $10\times130$ events with $5< Q^{2}_{DA} < 6$~GeV$^{2}$. One bin wide slices along the $x$-axis are taken from this figure. The RMS widths of these slices are plotted as a function of the bin centroid in $-t_{MC}$ in the (Right) figure. The $y$-axis can be interpreted as the resolution in $t$.}
\label{fig:tRes_10x130_Q2_5_6}
\end{figure}

\begin{figure}
    \centering
    \includegraphics[width=0.495\textwidth]{Figures/Appendix_Figs/10on130_tRes_2D_17_18_Q2.png}
    \includegraphics[width=0.495\textwidth]{Figures/Appendix_Figs/10on130_tRes_17_18_Q2}
    \caption{(Left) $\Delta t = t_{mc} - t_{eXBABE}$ as a function of $-t_{MC}$ for $10\times130$ events with $17< Q^{2}_{DA} < 18$~GeV$^{2}$. The RMS widths of these slices are plotted as a function of the bin centroid in $-t_{MC}$ in the (Right) figure. The $y$-axis can be interpreted as the resolution in $t$.}
\label{fig:tRes_10x130_Q2_17_18}
\end{figure}

\begin{figure}
    \centering
    \includegraphics[width=0.495\textwidth]{Figures/Appendix_Figs/10on130_tRes_2D_32_33_Q2.png}
    \includegraphics[width=0.495\textwidth]{Figures/Appendix_Figs/10on130_tRes_32_33_Q2}
    \caption{(Left) $\Delta t = t_{mc} - t_{eXBABE}$ as a function of $-t_{MC}$ for $10\times130$ events with $32< Q^{2}_{DA} < 33$~GeV$^{2}$. The RMS widths of these slices are plotted as a function of the bin centroid in $-t_{MC}$ in the (Right) figure. The $y$-axis can be interpreted as the resolution in $t$.}
\label{fig:tRes_10x130_Q2_32_33}
\end{figure}

\begin{figure}
    \centering
    \includegraphics[width=0.495\textwidth]{Figures/Appendix_Figs/10on250_tRes_2D_5_6_Q2.png}
    \includegraphics[width=0.495\textwidth]{Figures/Appendix_Figs/10on250_tRes_5_6_Q2}
    \caption{(Left) $\Delta t = t_{mc} - t_{eXBABE}$ as a function of $-t_{MC}$ for $10\times250$ events with $5< Q^{2}_{DA} < 6$~GeV$^{2}$. One bin wide slices along the $x$-axis are taken from this figure. The RMS widths of these slices are plotted as a function of the bin centroid in $-t_{MC}$ in the (Right) figure. The $y$-axis can be interpreted as the resolution in $t$.}
\label{fig:tRes_10x250_Q2_5_6}
\end{figure}

\begin{figure}
    \centering
    \includegraphics[width=0.495\textwidth]{Figures/Appendix_Figs/10on250_tRes_2D_17_18_Q2.png}
    \includegraphics[width=0.495\textwidth]{Figures/Appendix_Figs/10on250_tRes_17_18_Q2}
    \caption{(Left) $\Delta t = t_{mc} - t_{eXBABE}$ as a function of $-t_{MC}$ for $10\times250$ events with $17< Q^{2}_{DA} < 18$~GeV$^{2}$. The RMS widths of these slices are plotted as a function of the bin centroid in $-t_{MC}$ in the (Right) figure. The $y$-axis can be interpreted as the resolution in $t$.}
\label{fig:tRes_10x250_Q2_17_18}
\end{figure}

\begin{figure}
    \centering
    \includegraphics[width=0.475\textwidth]{Figures/Appendix_Figs/10on250_tRes_2D_32_33_Q2.png}
    \includegraphics[width=0.475\textwidth]{Figures/Appendix_Figs/10on250_tRes_32_33_Q2}
    \caption{(Left) $\Delta t = t_{mc} - t_{eXBABE}$ as a function of $-t_{MC}$ for $10\times250$ events with $32< Q^{2}_{DA} < 33$~GeV$^{2}$. The RMS widths of these slices are plotted as a function of the bin centroid in $-t_{MC}$ in the (Right) figure. The $y$-axis can be interpreted as the resolution in $t$.}
\label{fig:tRes_10x250_Q2_32_33}
\end{figure}

For $10\times250$ events, the situation is more nuanced. Wider $t$ bins may be needed even at modest $Q^{2}$, such as in Fig.\ref{fig:tRes_10x130_Q2_17_18} or a potentially substantial bin migration systematic  may be needed. It is only at very low $Q^{2}$, such as in Fig.~\ref{fig:tRes_10x130_Q2_5_6}, that the $t$ resolution remains comfortably below the bin width for the $t$ range of relevance for the extraction of $F_{\pi}$.

\bibliographystyle{elsarticle-num} 
\bibliography{bibliography.bib}

\end{document}