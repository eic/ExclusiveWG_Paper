\documentclass[letterpaper,12pt]{article}
\usepackage[utf8]{inputenc}
\pagestyle{plain}
\usepackage{graphicx}
\usepackage[small]{caption}
\usepackage{cite}
\usepackage{amsmath}
\usepackage{multirow}
\usepackage{setspace}
\usepackage{xcolor}
\usepackage{authblk}
\usepackage{url}
\usepackage{hyperref}

\pagenumbering{roman}
%Insert your title and author list below
\title{Light Meson Form Factors from \textbf{D}eep \textbf{E}xclusive \textbf{M}eson \textbf{P}roduction in early EIC science configurations with the ePIC Detector}
\author[1]{G.M. Huber}
\author[2]{S.J.D. Kay}
\author[1]{L. Preet}

\affil[1]{Department of Physics, University of Regina, SK, S4S 0A2, Canada}
\affil[2]{School of Physics, Engineering and Technology University of York, YO10 5DD, UK}
\date{MONTH 2025}

\begin{document}

\maketitle
\begin{abstract}
Abstract goes here
\end{abstract}

\begin{figure}[h]
    \centering
    \includegraphics[scale=0.5]{Figures/EPIC-logo_black.png}
\end{figure}

\pagebreak
\tableofcontents

\pagebreak
\pagenumbering{arabic}

\section{Introduction}\label{sec:Intro}


Pions and kaons are among the most prominent strongly interacting particles next to the nucleon, since
they are the Goldstone bosons of QCD. Thus, it is important to study their internal structure and how this reflects their Goldstone boson
nature; a question particularly relevant for understanding the origin of mass generation in QCD.  

The hard contribution to the $\pi^+$ form factor can be calculated exactly
within the framework of pQCD, and at asymptotically high $Q^2$ it takes a
particularly simple form, $F_{\pi}(Q^2) \overrightarrow{_{Q^2 \rightarrow
\infty}} 16 \pi \alpha_s(Q^2)f_{\pi}^2/Q^2$ \cite{PETERLEPAGE1979359}, where $f_{\pi}$ is
the $\pi^+$ decay constant.  In general, the pion also contains soft
contributions, which are expected to dominate at lower $Q^2$.  The
actual behavior of $F_{\pi}$ as a function of $Q^2$, as QCD transitions smoothly
from the non-perturbative (long-distance scale) confinement regime to the
perturbative (short-distance scale) regime, is an important test of our
understanding of QCD in bound hadron systems.  Since QCD calculations cannot
yet be performed rigorously in the confinement regime, experimental data
from JLab play a vital role in validating the theoretical approaches employed.
In particular, due to the charged pion's relatively simple quark-antiquark ($q\bar{q}$)
valence structure and its
experimental accessibility, the pion elastic form factor ($F_{\pi}$) offers our best
hope of directly observing QCD's transition from color-confinement at
long distance scales to asymptotic freedom at short distances. 
It is worth highlighting that in QCD the difference between the kaon and pion charge form factors is of the scale of 20\% at $Q^2 \sim$ 5 GeV$^2$~\cite{Gao:2017mmp} and disappears at asymptotic $Q^2$ as ln($Q^2$).  Thus, the acquisition of experimental data for both form factors covering a wide $Q^2$ range should be a high priority.

Current experimental information
on the pion and kaon form factors is limited, particularly at large $Q^2$ \cite{Horn:2016rip}.
Measurement of the $\pi^+$ electromagnetic form factor for $Q^2>0.3$ GeV$^2$ can be accomplished by the detection of the exclusive reaction $p(e,e'\pi^+)n$ at low $-t$. 
This is best described as quasi-elastic ($t$-channel) scattering of the electron from the
virtual $\pi^+$ cloud of the proton, where $t=(p_{p}-p_{n})^2$ is the
Mandelstam momentum transfer to the target nucleon.
Scattering from the $\pi^+$ cloud dominates the longitudinal
photon cross section ($d\sigma_L/dt$),  when $|t|\ll m_p^2$.  To reduce background contributions, one preferably separates the
components of the cross section due to longitudinal (L) and transverse (T)
virtual photons (and the LT, TT interference contributions), via a Rosenbluth
separation.  
A
Rosenbluth separation involves the absolute subtraction of two measurements
determined at high- and low-virtual photon polarization ($\epsilon_{Hi}$,
$\epsilon_{Lo}$), corresponding to high and low electron beam energies, with
very different detector rates. The resulting errors on $\sigma_L$ and $\sigma_T$ are
magnified by $1/\delta\epsilon=(\epsilon_{Hi}-\epsilon_{Lo})^{-1}$.  To keep the uncertainties in
$\sigma_L$ to an acceptable level, $\delta\epsilon >0.2$ is typically required,
{\it i.e.} an uncertainty magnification of no more than 5.  
The measurements
require continuous, high intensity electron beams, and detectors
with good particle identification and reproducible systematics.
JLab Hall~C is currently the only facility worldwide capable of such studies.  

At the EIC, $\pi^+$ form factor measurements can be extended to significantly
larger $Q^2$ than possible at JLab.  We have written an exclusive $p(e,e'\pi^+n)$ event
generator~\cite{DEMPgen} and performed detailed simulations to determine the
feasibility of $F_{\pi}$ measurements at the EIC\@.  The key
questions we have addressed include: 1) detector requirements to cleanly identify exclusive $e' \pi^+ n$ coincidences; 2) experimental acceptance and projected counting rates for such triple
coincidences; 3) event reconstruction resolution requirements to reliably extract $F_{\pi}(Q^2)$ from $p(e,e'\pi^+n)$ data.
Since the cross section falls
rapidly as the distance from the pion pole $(t-m_{\pi}^2)$ is increased, this
steep fall off needs to be measured to confirm the dominance of the pion cloud
mechanism in the acquired data.  
This note describes our work addressing all of these questions.


\section{Simulation Overview}\label{sec:Sim_Overview}

Details on the simulation, event generator used, detector geometry, beam conditions etc. Details on the relevant simulation campaign. Reference other sections via Sec.~\ref{sec:Intro} and cite things via \cite{jr:2022_YR}.

\subsection{Event Generator Details}\label{subsec:EvGen}

Information on event generator utilised. Specify version, link to instructions for running code (should be provided for Production WG). Break out other components into subsections if desired/needed (e.g. if simulation geometry is specific/unique).

\section{Event Selection}\label{sec:EvSelect}

Information on your event selection procedure/process. Highlight any specific subsystems used if relevant. Outline cuts used, ordering, rationale. If a cut is ``unusual'' or non-standard, make sure you discuss it. When showing kinematic variables, \textbf{clarify the reconstruction method.}. If needed, include a description of the reconstruction method.

\subsection{Analysis Code}\label{subsec:Analysis_Code}

Information on where to find your analysis code, how to run it. Consider adding additional details to the appendix (weird compilation quirks etc).

\section{Results and Discussion}\label{sec:Results_Discuss}

Your results with key performance plots etc. Consult checklist of key figures, see \href{https://docs.google.com/presentation/d/1bqz9_GPvPoW4oz1m8KvzuUhPJZBe_CfU5APMt0LjfaU/edit?slide=id.g3338e3f4b69_0_51#slide=id.g3338e3f4b69_0_51}{page three of this presentation} as an example of plots to include.


Once triple-coincidence $p(e,e'\pi^+n)$ events are cleanly identified with ePIC,
the value of $F_{\pi}(Q^2)$ is determined by
comparing the measured $d\sigma/dt$ values at small $-t$ to the best
available electroproduction model.  The obtained $F_{\pi}$ values are in
principle dependent upon the model used, but one anticipates this dependence to
be reduced at sufficiently small $-t$.
Measurements over a range of $-t$ are an essential part of the model validation process.
The JLab 6\,GeV experiments were instrumental in establishing the reliability of this technique up to $Q^2=2.45$~GeV$^2$~\cite{Huber:2008id, Horn:2016rip, Horn:2007ug, Volmer:2000ek, Horn:2006tm, Tadevosyan:2007yd, Blok:2008jy, Huber:2014ius, Huber:2014kar}, and extensive further tests are planned as part of JLab experiment E12-19-006 \cite{E12-19-006}.

{\bf GH can add more discussion later.}

\pagebreak
\appendix
\section{Appendix}
\label{Appendix:tRes}

As discussed in Sec.~\ref{sec:statistics}, the resolution in $t$ varies with beam energy and across $Q^{2}$ bins, as shown in Figs.~\ref{fig:tRes_10x130_Q2_5_6}--\ref{fig:tRes_10x250_Q2_32_33}. For $10\times130$ events, the resolution in $t$ is comfortably within the chosen width of each bin for the range of $t$ that is used in the form factor determination. The exception to this is for high $-t$ values in higher $Q^{2}$ bins, as can be seen in Fig.~\ref{fig:tRes_10x130_Q2_32_33}. As such, for the high $Q^{2}$ bins at this beam energy, either broader $t$ bins must be used, or a bin-migration systematic will need to be applied. 

\begin{figure}[h]
    \centering
    \includegraphics[width=0.495\textwidth]{Figures/Appendix_Figs/10on130_tRes_2D_5_6_Q2.png}
    \includegraphics[width=0.495\textwidth]{Figures/Appendix_Figs/10on130_tRes_5_6_Q2}
    \caption{(Left) $\Delta t = t_{mc} - t_{eXBABE}$ as a function of $-t_{MC}$ for $10\times130$ events with $5< Q^{2}_{DA} < 6$~GeV$^{2}$. One bin wide slices along the $x$-axis are taken from this figure. The RMS widths of these slices are plotted as a function of the bin centroid in $-t_{MC}$ in the (Right) figure. The $y$-axis can be interpreted as the resolution in $t$.}
\label{fig:tRes_10x130_Q2_5_6}
\end{figure}

\begin{figure}
    \centering
    \includegraphics[width=0.495\textwidth]{Figures/Appendix_Figs/10on130_tRes_2D_17_18_Q2.png}
    \includegraphics[width=0.495\textwidth]{Figures/Appendix_Figs/10on130_tRes_17_18_Q2}
    \caption{(Left) $\Delta t = t_{mc} - t_{eXBABE}$ as a function of $-t_{MC}$ for $10\times130$ events with $17< Q^{2}_{DA} < 18$~GeV$^{2}$. The RMS widths of these slices are plotted as a function of the bin centroid in $-t_{MC}$ in the (Right) figure. The $y$-axis can be interpreted as the resolution in $t$.}
\label{fig:tRes_10x130_Q2_17_18}
\end{figure}

\begin{figure}
    \centering
    \includegraphics[width=0.495\textwidth]{Figures/Appendix_Figs/10on130_tRes_2D_32_33_Q2.png}
    \includegraphics[width=0.495\textwidth]{Figures/Appendix_Figs/10on130_tRes_32_33_Q2}
    \caption{(Left) $\Delta t = t_{mc} - t_{eXBABE}$ as a function of $-t_{MC}$ for $10\times130$ events with $32< Q^{2}_{DA} < 33$~GeV$^{2}$. The RMS widths of these slices are plotted as a function of the bin centroid in $-t_{MC}$ in the (Right) figure. The $y$-axis can be interpreted as the resolution in $t$.}
\label{fig:tRes_10x130_Q2_32_33}
\end{figure}

\begin{figure}
    \centering
    \includegraphics[width=0.495\textwidth]{Figures/Appendix_Figs/10on250_tRes_2D_5_6_Q2.png}
    \includegraphics[width=0.495\textwidth]{Figures/Appendix_Figs/10on250_tRes_5_6_Q2}
    \caption{(Left) $\Delta t = t_{mc} - t_{eXBABE}$ as a function of $-t_{MC}$ for $10\times250$ events with $5< Q^{2}_{DA} < 6$~GeV$^{2}$. One bin wide slices along the $x$-axis are taken from this figure. The RMS widths of these slices are plotted as a function of the bin centroid in $-t_{MC}$ in the (Right) figure. The $y$-axis can be interpreted as the resolution in $t$.}
\label{fig:tRes_10x250_Q2_5_6}
\end{figure}

\begin{figure}
    \centering
    \includegraphics[width=0.495\textwidth]{Figures/Appendix_Figs/10on250_tRes_2D_17_18_Q2.png}
    \includegraphics[width=0.495\textwidth]{Figures/Appendix_Figs/10on250_tRes_17_18_Q2}
    \caption{(Left) $\Delta t = t_{mc} - t_{eXBABE}$ as a function of $-t_{MC}$ for $10\times250$ events with $17< Q^{2}_{DA} < 18$~GeV$^{2}$. The RMS widths of these slices are plotted as a function of the bin centroid in $-t_{MC}$ in the (Right) figure. The $y$-axis can be interpreted as the resolution in $t$.}
\label{fig:tRes_10x250_Q2_17_18}
\end{figure}

\begin{figure}
    \centering
    \includegraphics[width=0.475\textwidth]{Figures/Appendix_Figs/10on250_tRes_2D_32_33_Q2.png}
    \includegraphics[width=0.475\textwidth]{Figures/Appendix_Figs/10on250_tRes_32_33_Q2}
    \caption{(Left) $\Delta t = t_{mc} - t_{eXBABE}$ as a function of $-t_{MC}$ for $10\times250$ events with $32< Q^{2}_{DA} < 33$~GeV$^{2}$. The RMS widths of these slices are plotted as a function of the bin centroid in $-t_{MC}$ in the (Right) figure. The $y$-axis can be interpreted as the resolution in $t$.}
\label{fig:tRes_10x250_Q2_32_33}
\end{figure}

For $10\times250$ events, the situation is more nuanced. Wider $t$ bins may be needed even at modest $Q^{2}$, such as in Fig.\ref{fig:tRes_10x130_Q2_17_18} or a potentially substantial bin migration systematic  may be needed. It is only at very low $Q^{2}$, such as in Fig.~\ref{fig:tRes_10x130_Q2_5_6}, that the $t$ resolution remains comfortably below the bin width for the $t$ range of relevance for the extraction of $F_{\pi}$.

\bibliographystyle{elsarticle-num} 
\bibliography{bibliography.bib}

\end{document}